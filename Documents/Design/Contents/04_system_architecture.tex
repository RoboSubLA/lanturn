% ------------------------------------------------------------------------------
% System Architecture
% ------------------------------------------------------------------------------
\section{System Architecture}
\label{sec:architecture}

The software on Lanturn is organized into a number of modules, each of which is
responsible for a specific task. These modules are build on the ROS framework
and communicate with each other using ROS messages.

% ------------------------------------------------------------------------------
% Overview of the System
% ------------------------------------------------------------------------------
\subsection{Overview of the System}
\label{sec:overview}
 
Computer Vision shall output to: Mapping, Localization. Mapping shall output
to: Autonomy, Localization. Localization shall output to: Autonomy Autonomy
shall output to: Controls. Controls shall output to: Mapping, Localization.
\par

Autonomy will manage time, manage state machines, read and interpret
mapping/localization data, read and filter computer vision data, control claw,
shoot torpedoes, release dropper,  autonomously navigate map, and
position/orient/center to desired orientations. 
\par

Comp Vision will publish raw images from front and bottom cams, detect and
classify all task objects, provide distance from objects, calculate angle of
incidence of objects. 
\par

Controls will read and publish data from Bar30 barometer, VN-100 IMU, Teledyne
DVL, Sonar. Controls will also implement a PID library, generate PWM values
that will move the sub to desired position, output PWM values to thrusters.
Controls will also control a mechanical claw, shoot torpedoes, and release a
ball from dropper all on command. 
\par

Mapping will subscribe to all computer vision data and Sonar data. Mapping will
also implement a Kalman Filter, generate a map of the environment, position all
task objects in map, and publish map. 
\par

Localization will subscribe to IMU data topic, Barometer data topic, both
camera topics, DVL data topic, and map topic. Localization will position and
orient the sub inside the map, and publish localization data.
\par

% ------------------------------------------------------------------------------
% Autonomy
% ------------------------------------------------------------------------------
\subsection{Autonomy}
\label{sec:autonomy}

The autonomy part of the software is responsible for completing the robosub
goals. This is done using a state machine and using the ROS software to
navigate the data to move from one state to another until each goal is
completed, battery runs out, or time limit of 20 minutes is exceeded. The state
machines are subscribed to all hardware components to be able to determine
ubliches the next goal, and to controls to reach the next goal or complete the
task. Each task/goal is broken down into their own states in their own class.
Future implementation will move from state machine to behavior tree which
allows for system to be goal oriented instead.
\par

% ------------------------------------------------------------------------------
% Computer Vision
% ------------------------------------------------------------------------------
\subsection{Computer Vision}
\label{sec:computer_vision}

The DFD Level 1 will start with the Camera sending information to the Image
Processor, after processing the image it will send the information to Object
Detection. Object Detection will process that information and once it’s able to
detect the object it will send the data to Object Classification, and it will
define the location of the object. Once Object Classification processes the
data of the object detected, it will send a Message Output. The Message Output
will give out the data of the Object ID and the Bounding Box.
\par

% ------------------------------------------------------------------------------
% Controls
% ------------------------------------------------------------------------------
\subsection{Controls}
\label{sec:controls}

The controls code has the same form as any other microcontroller code. It
starts with a setup() function then starts executing a loop() function until
told to otherwise. 
\par

In the setup() function, sensors, actuators and the thruster motors are
initialized and general setup like importing libraries and configuring the PID
controller is done here. This function is executed once then control is turned
over to the loop() function. 
\par

The loop() function contains the code for the PID controller, sensor reading
and actuator control that will run independently. The flow of the data is this:
grab sensor data, fix setpoints to account for circular rollover error, compute
PWM values, combine the PWM values, output them to the ESCs and loop back to
the beginning.
\par

% ------------------------------------------------------------------------------
% Localization
% ------------------------------------------------------------------------------
\subsection{Localization}
\label{sec:localization}

The localization node takes input data from the various environmental sensors
(IMU, Sonar, Barometer, DVL, hydrophones). Localization will process this data
and then publish the processed data to the entire system. Any node that
subscribes will have access to the data.
\par

% ------------------------------------------------------------------------------
% Mapping
% ------------------------------------------------------------------------------
\subsection{Mapping}
\label{sec:mapping}

TBD.
\par

