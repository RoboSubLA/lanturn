% ------------------------------------------------------------------------------
% Design Considerations
% ------------------------------------------------------------------------------
\section{Design Considerations}
\label{sec:design}

This section describes many of the issues which need to be addressed or
resolved before attempting to devise a complete design solution.
\par

% ------------------------------------------------------------------------------
% Assumptions and Dependencies
% ------------------------------------------------------------------------------
\subsection{Assumptions and Dependencies}
\label{sec:assumptions}

Lanturn will have a TX2 module, a small form-factor embedded computing device,
for its main computer. The TX2 module comes flashed with Ubuntu 18.04 and comes
packaged with Jetpack 4.6.1, a set of libraries designed for AI computations
designed for the TX2 module.  
\par

Over the Operating System base and the Kernel overlay will be ROS2 Foxy, a
piece of middleware that sets an environment for the different processes that
will run on Lanturn. 
\par

The second computing device that will exist in the Lanturn system is a
microcontroller, a teensy 4.1. The microcontroller will be used to interface
with the onboard sensors, motors, and actuators.
\par

% ------------------------------------------------------------------------------
% General Constraints
% ------------------------------------------------------------------------------
\subsection{General Constraints}
\label{sec:constraints}

Ubuntu is the only operating system that can run on the TX2 module. This
requires any drivers or other implemented interfaces to be designed for, or to
be compatible with Ubuntu versions. 
\par

The TX2 module and its accompanying software is designed for Ubuntu 18.04. Even
though it is designed for Ubuntu version 18.04, it is possible to upgrade to a
later Ubuntu version; however, the only Ubuntu version which is officially
supported is Ubuntu 18.04. The Ubuntu version the TX2 will be running on is
Ubuntu 20.04 Focal; all constraints that come with running on an unsupported
version of Ubuntu on the TX2 module will exist in this project. 
\par

The middleware running on top of Ubuntu is a ROS2 version, ROS2 Foxy, which
runs on Ubuntu 20.04. ROS2 Foxy comes with restrictions with the programming
languages that can be used. The API that is used for development are rclcpp
(C++ library) and rclpy (Python library). 
\par

The second computing device, Teensy 4.1, must be programmed in C++ using one
of: teensyduino with Arduino IDE or platformio for Visual Studio Code.
\par

% ------------------------------------------------------------------------------
% Goals and Guidelines
% ------------------------------------------------------------------------------
\subsection{Goals and Guidelines}
\label{sec:goals}

This document will have the following goals and guidelines in mind: 

\begin{itemize}
    \item The Lanturn project must have a functioning autonomous system that
        can execute competition tasks by July 2023.
    \item All code written for this project should be documented. 
    \item Programs written for this project should be refactored periodically
        to improve efficiency.
\end{itemize}

% ------------------------------------------------------------------------------
% Development Methodology
% ------------------------------------------------------------------------------
\subsection{Development Methodology}
\label{sec:methodology}

The software team will be broken down into 5 sub-teams, each responsible for
one module of the software system. Each week, the sub-teams show progress made
and communicate any difficulties that have come across. 
\par

As sub-teams gain deeper understanding of their module, they will communicate
to the other modules what data they can provide and what data they cannot. 
\par

This system will ensure that none of the modules depend on each other and can
be switched out, if need be, in future versions of Lanturn.
\par

