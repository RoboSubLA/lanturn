% ------------------------------------------------------------------------------
% Overall Description
% ------------------------------------------------------------------------------
\section{Overall Description}
\label{sec:overall}

The requirements for the software developed for Lanturn will be largely derived
from version 3 of the 2022 RoboSub Mission Handbook. The mission handbook
states the logistics of the competition, the rules during the competition and
the tasks to be done by any AUV participating.  
\par

Along with the requirements derived from the 2022 Mission Handbook, additional
requirements will be added based on what was learned from previous years and
from suggestions from the team’s advisor(s). 
\par

As of the publish date of this document, the 2023 Mission Handbook has not been
released. In later versions of this document, the requirements will be updated
for every version released of the 2023 Mission Handbook.
\par

% ------------------------------------------------------------------------------
% System Analysis
% ------------------------------------------------------------------------------
\subsection{System Analysis}
\label{sec:analysis}

The goal of the project is to develop software that will accomplish as many
tasks as possible as quickly as possible with the sensors and actuators
available on the submarine.

\begin{enumerate}
    \item Problems
        \begin{itemize}
            \item 100\% autonomous operation
            \item Submodule checks
            \item Data accuracy
            \item Testing Limitations
            \item Self-diagnosis
            \item Lanturn manufacturing time
        \end{itemize}
    \item Solutions
        \begin{itemize}
            \item Behavior Trees
            \item Global watchdog service
            \item Local watchdog services
            \item Sensor fusion
            \item Simulation
        \end{itemize}
\end{enumerate}

% ------------------------------------------------------------------------------
% Product Perspective
% ------------------------------------------------------------------------------
\subsection{Product Perspective}
\label{sec:perspective}

The RoboSub Organization states the following on their website: “The behaviors
demonstrated by these experimental AUVs mimics those of real-world systems,
currently deployed around the world for underwater exploration, seafloor
mapping, and sonar localization, amongst many others.” [1] 
\par

Since Lanturn is to be completely autonomous, the system itself is completely
isolated from all other systems, an exception is when it is tethered to an
external computer for real-world testing and (re)configuration. 
\par

Concepts used by all or some modules in the AUV can be applied to computer
networks, other robotics projects, surveillance systems, ROVs (remote operated
vehicle), UAV (Unmanned aerial vehicles), video-game programs (programming
behavior of NPCs), electronics projects, systems designed for underwater
exploration (or general exploration of territory), systems designed for
planetary exploration, cartography, localization, any system that will benefit
from more accurate sensor data, agriculture (autonomous behavior) and other
software systems.
\par

% ------------------------------------------------------------------------------
% Product Functions
% ------------------------------------------------------------------------------
\subsection{Product Functions}
\label{sec:functions}

Functionality of the software will be derived from the requirements of the
2022 RoboSub Mission Handbook. The software will be designed to accomplish the
tasks listed in the mission handbook.

\begin{enumerate}
    \item With Moxy (Coin Flip) 
    \item Choose Your Side (Gate) 
    \item Collecting (Bins) 
    \item Make the Grade (Buoys) 
    \item Survive the Shootout (Torpedoes) 
    \item Cash or Smash (Octagon) 
    \item Follow Orange Guide Markers
\end{enumerate}

None of the functions (tasks) laid out in the Mission Handbook are required
(except Choose Your Side), however, the more tasks the AUV can do, the more
points will be scored for the mission. The assumption that will be made, for
now, is that all tasks will at least be attempted.
\par

Other functions that will be implemented are:
\begin{itemize}
    \item Autonomous navigation
    \item Autonomous decision making
    \item Object detection
    \item Object classification
    \item Object relative position estimation
    \item Vehicle localization
    \item Vehicle state estimation
    \item Vehicle state control
    \item Vehicle state monitoring
    \item Vehicle state logging
    \item Environment mapping
\end{itemize}

The functionality listed above is all the general functionality that will
enable Lanturn to accomplish any one of the competition tasks.
\par

% ------------------------------------------------------------------------------
% User Classes and Characteristics
% ------------------------------------------------------------------------------
\subsection{User Classes and Characteristics}
\label{sec:classes}

% define column types
\newcolumntype{P}[1]{>{\centering\arraybackslash}p{#1}} % Horizontal Cell Padding

% define table
\begin{table}[htbp]
{
    \renewcommand{\arraystretch}{1.5} % Vertical Cell Padding
    \begin{tabularx}{\textwidth}{| P{6em} | X |}
        \hline
        \textbf{User Class} & \textbf{Description} \\
        \hline
        Swimmer     & Will start Lanturn and monitor its state. \\
        \hline
        Tester      & Will verify that Lanturn is working as expected. \\
        \hline
        Developer   & Will develop the software for Lanturn. \\
        \hline
    \end{tabularx}
    \caption{\label{tab:classes} User Classes and Characteristics}
}
\end{table}

% ------------------------------------------------------------------------------
% Operating Environment
% ------------------------------------------------------------------------------
\subsection{Operating Environment}
\label{sec:environment}

The hardware that will be used for the development of Lanturn is the following:

\textbf{TX2 Module with Development Kit Carrier Board}
\begin{itemize}
    \item Ubuntu 20.04
    \item NVIDIA JetPack 4.5
    \item NVIDIA CUDA 10.2
    \item NVIDIA cuDNN 8.0
    \item NVIDIA TensorRT 7.1
    \item ROS2 Foxy Fitzroy
    \item OpenCV 4.5.1
    \item Python 3.8.10+
    \item C++ 17
\end{itemize}

The software will be developed on a computer running Ubuntu 20.04 and using the
ROS2 framework.
\par

\textbf{Teensy 4.1}
\begin{itemize}
    \item PlatformIO
    \item Arduino Framework
    \item Micro-ros-plarformio
    \item C++ 17
\end{itemize}

% ------------------------------------------------------------------------------
% Design and Implementation Constraints
% ------------------------------------------------------------------------------
\subsection{Design and Implementation Constraints}
\label{sec:constraints}

All software (except interfaces with sensors and actuators) must be implemented
in Ubuntu 20.04, as that is the only operating system that is supported by the
Nvidia TX2 module. By extension, any software designed to be run in the TX2
module will also have to be a ROS2 package. 
\par

All software designed to interface with sensors or actuators must be compatible
with communication protocols available for the Teensy 4.1 microcontroller. 
\par

All software designed must follow the limitations set by the policies listed in
the Mission Handbook provided by RoboNation. 
\par

If, the ME/EE team for RoboSub cannot design or manufacture any of the
actuators, that is, the claw, torpedoes, or dropper, then the software team
will not be able to test or implement any software designed to execute any
functionality requiring missing actuator(s).
\par

% ------------------------------------------------------------------------------
% User Documentation
% ------------------------------------------------------------------------------
\subsection{User Documentation}
\label{sec:documentation}

Each sub-module of the system will come with documentation in the form of a
README.md on the project’s GitHub repository. 
\par

The structure of the README’s will be as follows: 

\begin{enumerate}
    \item Overview: An overview of the role played by the submodule 
    \item Dependencies: A list of dependencies required to develop and run the sub-module 
    \item Installation: Instructions on how to install any software needed for the submodule 
    \item Setup: Instructions on how to setup the environment for either development or deployment 
    \item Usage: Instructions on how to use the submodule on the submarine and on the field 
    \item Problems and Fixes: Common issues encountered and, either, how to fix or work around the issue 
\end{enumerate}

% ------------------------------------------------------------------------------
% Assumptions and Dependencies
% ------------------------------------------------------------------------------
\subsection{Assumptions and Dependencies}
\label{sec:assumptions}

The assumption while developing the software for this system is that the rules
for the 2023 RoboSub competition will largely remain the same as the previous
year.

% ------------------------------------------------------------------------------
% Apporitionment of Requirements
% ------------------------------------------------------------------------------
\subsection{Apportionment of Requirements}
\label{sec:apportionment}

New software requirements may arise from the release of the 2023 RoboSub
Mission Handbook, and it is the responsibility of the team to adjust the
software system to any new requirement.

