% ------------------------------------------------------------------------------
% External Interface Requirements
% ------------------------------------------------------------------------------
\section{External Interface Requirements}
\label{sec:external}

This software does not currently have any custom interfaces, but there are
interfaces provided by ROS2 which can be used to configure or control the
submarine in real time.
\par

% ------------------------------------------------------------------------------
% User Interfaces
% ------------------------------------------------------------------------------
\subsection{User Interfaces}
\label{sec:user}

Currently, there are not any graphical user interfaces designed for interfacing
with Lanturn. However, ROS provides a general way of interfacing and
interpreting data from a ROS system. 
\par

\subsubsection*{RQT}
\label{sec:rqt}

RQT is a software framework for creating graphical user interfaces with the
running ROS system. It behaves like a module within the ROS system and comes
with implementation of common functionality in the form of plugins.
\par

Above is a sample configuration of several plugins that might be useful to have
when interpreting data from Lanturn’s ROS system. 
\par

Of note is the drop-down menus at the top left. The drop-down menu shown when
selecting “Plugins” gives a list of plugins to open. Once a plugin is selected,
it will open in the RQT window and then can be dragged around to reconfigure
the layout of the plugins. 
\par

More information can be found on their website http://wiki.ros.org/rqt.
\par

\subsubsection*{RVIZ}
\label{sec:rviz}

Rviz is a 3D visualization tool for ROS that can be used to visualize aspects
of a robot model. It provides different panels that can help you visualize
position, orientation and camera data in the running ROS system.
\par

More information can be found on their website http://wiki.ros.org/rviz.
\par

% ------------------------------------------------------------------------------
% Hardware Interfaces
% ------------------------------------------------------------------------------
\subsection{Hardware Interfaces}
\label{sec:hardware}

\subsubsection*{Fathon-X}
\label{sec:fathon-x}

Although Lanturn is meant to be entirely autonomous, it is beneficial to be
able to connect to the ROS system while testing new components for fast testing
and real-time development. 
\par

Lanturn will have a Fathom-X, an interface board sold by Blue Robotics, onboard
that can be used to establish a connection to an external computer. To
establish the connection, two of the Fathom-X boards are required: one onboard
Lanturn and the other connected to the external computer through an Ethernet
port.
\par

\textbf{Fathon-X Features}
\begin{itemize}
    \item 80 Mbps Ethernet over two wires (per our own bandwidth testing) 
    \item 300m+ tether length capability 
    \item Plug-and-play with no setup involved 
    \item Onboard switching power supply with 7-28V input range 
    \item USB Mini-B connector for powering directly from a computer on the topside 
    \item Indicator LEDs for power, link, and data 
    \item Included 6" Ethernet cable for connection to onboard computer
\end{itemize}

More on the Blue Robotics website:
https://bluerobotics.com/store/comm-control-power/tether-interface/fathom-x-tether-interface-board-set-copy/
\par

\textbf{Fathom ROV Tether}
The Fathom ROV Tether is another product from Blue Robotics and can be
connected to one of the penetrators on Lanturn to host the connection between
the two Fathom-X Boards.
\par

% ------------------------------------------------------------------------------
% Software Interfaces
% ------------------------------------------------------------------------------
\subsection{Software Interfaces}
\label{sec:software}

The external computer that is connected through the Tether should be running on
the same software as Lanturn. 

\begin{itemize}
    \item Ubuntu 20.04 
        \begin{itemize}
            \item ROS2 Foxy (Middleware) 
            \item OpenSSH
        \end{itemize}
\end{itemize}

% ------------------------------------------------------------------------------
% Communications Interfaces
% ------------------------------------------------------------------------------
\subsection{Communications Interfaces}
\label{sec:communications}

The external computer will use OpenSSH to establish an SSH connection with
Lanturn and will use the same Message Passing interfaces as the ROS system in
Lanturn (i.e., ROS CLI tools and ROS APIs).

