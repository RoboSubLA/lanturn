% -----------------------------------------------------------------------------
% Purpose
% -----------------------------------------------------------------------------
\section{Introduction}
\label{sec:introduction}

The computer science senior design team for RoboSub will work in conjunction
with another team of 10 consisting of undergraduates from the Mechanical and
Electrical Department of California State University, Los Angeles. The goal of
the two teams is to design, manufacture and program an AUV (Autonomous
Underwater Vehicle) to compete in the annual RoboSub [2] competition hosted by
RoboNation \cite{robonation_2022}.
\par

There are common requirements across previous RoboSub competitions; these
requirements include: autonomous navigation, movement, object detection, object
classification, and object manipulation. 
\par

This year, the RoboSub senior design team inherited the design of the previous
year’s senior design team. They started the mechanical, electrical and software
design for the Autonomous Underwater Vehicle: Lanturn. The RoboSub Computer
Science team will work with the Mechanical and Electrical Senior Design team to
finish the work done by the previous year. 
\par

This document will give an overview to both users and developers of the
expected behavior and software requirements for Lanturn.
\par

% -----------------------------------------------------------------------------
% Purpose 
% -----------------------------------------------------------------------------
\subsection{Purpose}
\label{sec:purpose}
\subsubsection{Purpose of the Document}

This document will list the software requirements imposed by RoboNation in the
2023 RoboSub competition and any self-imposed requirements. Self-imposed
requirements might include requirements for development, design improvements,
system modularity, or quality of life improvements.
\par

\begin{itemize}
    \item Section 3 will give requirements for setting up an external interface 
    \item Section 2 will give an overall description of the requirements 
    \item Section 4 will explicitly define specific requirements 
    \item Section 5 will define non-functional requirements 
    \item Section 6 will define any other requirement that does not fit in any other section 
\end{itemize}

\subsubsection{Purpose of the Product}

The product being developed is software for the Autonomous Underwater Vehicle,
Lanturn. Lanturn is to compete in the 2023 RoboSub competition hosted by
RoboNation. The RoboSub 2022 competition was hosted at The University of
Maryland in their Olympic pool. Other years, the competition is hosted at the
TRANSDEC center in San Diego. Since the competition for 2022 was a success in
the University of Maryland, a similar environment is expected for the 2023
competition. 
\par

Lanturn should be able to autonomously navigate and execute tasks within a
course. The layout of the course and a description of the tasks is published by
RoboNation by the end of March of the year of the competition. For now, the
software requirements and assumptions on what tasks Lanturn will need to do
will be based on the 2022 Mission Handbook.
\par

% ------------------------------------------------------------------------------
% Audience
% ------------------------------------------------------------------------------
\subsection{Intended Audience and Reading Suggestions}
\label{sec:audience}

This software document is intended for users, developers, documentation writers
and testers. 

\begin{itemize}
    \item Documentation Writers 
        \begin{itemize}
            \item The 2022 RoboSub Team Handbook 
            \item section 4 to understand the requirements of the system and what documentation might exist in other places 
            \item if they are responsible for document one software submodule, they can read their own section 
            \item section 5 for requirements that do not fit in other categories
        \end{itemize}

    \item Developers 
        \begin{itemize}
            \item The RoboSub Software Design Document 
            \item The appropriate README in the git repository hosted on GitHub 
            \item section 4 to get a complete understanding of the complete system (even if they are only responsible for one software module) 
            \item section 5 to get an understanding of restrictions they should keep in mind while developing the software
        \end{itemize}

    \item Users 
        \begin{itemize}
            \item The 2022 RoboSub Team Handbook 
            \item section 3 to understand what is required to interface with Lanturn 
        \end{itemize}

    \item Testers 
        \begin{itemize}
            \item The 2022 RoboSub Team Handbook 
            \item section 3 to understand what is required to interface with Lanturn
            \item section 4 to understand the functionality that needs to be tested
        \end{itemize}
\end{itemize}

% ------------------------------------------------------------------------------
% Scope
% ------------------------------------------------------------------------------
\subsection{Product Scope}
\label{sec:scope}
The goal of the software for Lanturn is to autonomously accomplish tasks at the
RoboSub competition. 
\par

To accomplish the set goal, the software will have five modules: 
\begin{enumerate}
    \item Autonomy
        \begin{itemize}
            \item Navigation 
            \item Decision making 
        \end{itemize}

    \item Computer Vision 
        \begin{itemize}
            \item Object detection 
            \item Object classification
        \end{itemize}

    \item Controls 
        \begin{itemize}
            \item Sensor data acquisition 
            \item Actuator control
        \end{itemize}

    \item Mapping 
        \begin{itemize}
            \item Environment mapping 
        \end{itemize}

    \item Localization 
        \begin{itemize}
            \item Vehicle localization in map
        \end{itemize}
\end{enumerate}

% ------------------------------------------------------------------------------
% Definitions, Acronyms, and Abbreviations
% ------------------------------------------------------------------------------
\subsection{Definitions, Acronyms, and Abbreviations}
\label{sec:definitions}

A list of Definition, Acronyms, and Abbreviations can be found at the end of
this document in Appendix A: Glossary.
\par

% ------------------------------------------------------------------------------
% References
% ------------------------------------------------------------------------------
\subsection{References}
\label{sec:references}

